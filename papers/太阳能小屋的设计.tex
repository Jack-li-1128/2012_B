\documentclass{ctexart}

\usepackage{appendix}
\usepackage{listings}% 插入代码
\usepackage{xcolor} 
\usepackage{graphicx}% 插入表格/图片
\usepackage{booktabs} % 绘制表格
\usepackage{caption} % 标题
\usepackage{geometry}
\usepackage{array}
\usepackage{amsmath}
\usepackage{subfigure} % 插入图片
\usepackage{longtable}
\usepackage{abstract}% 摘要
\pagestyle{plain} % 页眉消失
\usepackage{setspace}
\usepackage{multirow}% 表格
\usepackage{diagbox}
\usepackage{enumerate}% 序号
\usepackage{float}% 固定图片或表格的位置
\usepackage{gensymb}
\usepackage{microtype}

\geometry{a4paper,left=2.5cm,right=2.5cm,top=2cm,bottom=2cm}% 页边距
\lstset{
    numbers=left, % 设置行号位置
    numberstyle=\tiny, % 设置行号大小
    keywordstyle=\color{blue}, % 设置关键字颜色
    commentstyle=\color[cmyk]{1,0,1,0}, % 设置注释颜色
    escapeinside=``, % 逃逸字符(1左面的键),用于显示中文
    breaklines, % 自动折行
    extendedchars=false, % 解决代码跨页时,章节标题,页眉等汉字不显示的问题
    xleftmargin=1em,xrightmargin=1em, aboveskip=1em, % 设置边距
    tabsize=4, % 设置tab空格数
    showspaces=false % 不显示空格
}

\title{太阳能小屋的设计}
\date{}
\author{}

\begin{document}
    \maketitle
    \renewcommand{\abstractname}{\Large 摘要\\}
    \begin{abstract}
        \normalsize
        本文针对太阳能小屋设计问题,建立了
        
        针对问题一,为达到最高收益,最优的太阳能电池板选择,墙面选择,铺设方式选择与变压器选择,最终得到了最高的发电量,与最大经济效益。
        
        针对问题二,
        
        针对问题三,
        
        针对问题四,
        
        \textbf{关键字}:
    \end{abstract}
    \newpage
% 重新设置页面边距
    \newgeometry{a4paper,left=3.18cm,right=3.18cm,top=2.54cm,bottom=2.54cm}
	\section{问题背景与重述}
	\subsection{问题背景}
    在设计太阳能小屋时,需在建筑物外表面(屋顶及外墙)铺设光伏电池,光伏电池组件所产生的直流电需要经过逆变器转换成220V交流电才能供家庭使用,并将剩余电量输入电网。不同种类的光伏电池每峰瓦的价格差别很大,且每峰瓦的实际发电效率或发电量还受诸多因素的影响,如太阳辐射强度、光线入射角、环境、建筑物所处的地理纬度、地区的气候与气象条件、安装部位及方式(贴附或架空)等。因此,在太阳能小屋的设计中,研究光伏电池在小屋外表面的优化铺设是很重要的问题。
    
    在求解每个问题时,都要求配有图示,给出小屋各外表面电池组件铺设分组阵列图形及组件连接方式(串、并联)示意图,也要给出电池组件分组阵列容量及选配逆变器规格列表。在同一表面采用两种或两种以上类型的光伏电池组件时,同一型号的电池板可串联,而不同型号的电池板不可串联。在不同表面上,即使是相同型号的电池也不能进行串、并联连接。应注意分组连接方式及逆变器的选配。
    
    根据附件 1 提供的信息,附件 2 和附件 3 提供的,以及附件 4 提供的,可以建立数学模型解决以下问题:
    
    \subsection{问题表述}
    \begin{enumerate}[(1)]
        \item 问题一:请根据山西省大同市的气象数据,仅考虑贴附安装方式,选定光伏电池组件,对小屋(见附件2)的部分外表面进行铺设,并根据电池组件分组数量和容量,选配相应的逆变器的容量和数量。
        \item 问题二:电池板的朝向与倾角均会影响到光伏电池的工作效率,请选择架空方式安装光伏电池,重新考虑问题1。
        \item 问题三:根据附件7给出的小屋建筑要求,请为大同市重新设计一个小屋,要求画出小屋的外形图,并对所设计小屋的外表面优化铺设光伏电池,给出铺设及分组连接方式,选配逆变器,计算相应结果。
    \end{enumerate}

    \section{问题分析}
    \subsection{问题一分析}
    对于问题一进行数据预处理,为了达成年太阳能光伏发电总量尽可能大,而单位发电量的费用尽可能小这两个目标,我们一步步地对
    \subsection{问题二分析}
    首先通过,设计相应的架空方式
    \subsection{问题三分析}
    对于该问题,首先需要建立,需要设计接收阳光最优化的房子

    \section{模型假设}
    \begin{enumerate}[(1)]
        \item 在时间分布上,辐射度是连续的。
        \item 太阳
        \item 基于
        \item 假设
    \end{enumerate}

    \section{符号说明}
    \begin{center}
        \setlength{\tabcolsep}{9mm}{
            \begin{tabular}{ccc}
                \toprule  % 添加表格头部粗线。
                \textbf{符号} & \textbf{意义} & \textbf{单位}\\
                \midrule  % 添加表格中横线
                \textbf{K} & \textbf{聚类类别数} & \textbf{}\\
                \textbf{W} & \textbf{意义} & \textbf{单位}\\
                \textbf{i} & \textbf{意义} & \textbf{}\\
                \textbf{j} & \textbf{意义} & \textbf{}\\
                \textbf{符号} & \textbf{意义} & \textbf{单位}\\
                \textbf{符号} & \textbf{意义} & \textbf{单位}\\
                \textbf{符号} & \textbf{意义} & \textbf{单位}\\
                \textbf{符号} & \textbf{意义} & \textbf{单位}\\
                \textbf{符号} & \textbf{意义} & \textbf{单位}\\
                \textbf{符号} & \textbf{意义} & \textbf{单位}\\
                \textbf{符号} & \textbf{意义} & \textbf{单位}\\
                \bottomrule % 添加表格底部粗线
        \end{tabular}}
    \end{center}

    \section{模型建立与求解}
    \subsection{问题一模型的建立与求解}
    因为
    \subsubsection{问题一模型的建立}
    针对

    
    为解决辐射量的计算问题,对光能进行积分
    
    为解决最优太阳能电池板问题,

    为解决太阳能电池板排布问题,建立了

    为解决变压器问题,
    

    
    \subsubsection{数据预处理}
    拟合
    \subsubsection{最优太阳能电池板排布问题}

    \begin{figure}[H] % [H] 表示强制当前位置插入图片
        \centering % 使图片居中
        \includegraphics[width=0.8\textwidth]{"south_top_B2_21_C1_2_C6_63.png"
} % 图片文件名及宽度调整
        \caption{这是一个图片示例} % 图片标题
        \label{fig:example} % 图片标签
    \end{figure}

    \begin{figure}[H] % [H] 表示强制当前位置插入图片
        \centering % 使图片居中
        \includegraphics[width=0.5\textwidth]{example-image-a} % 图片文件名及宽度调整
        \caption{这是一个图片示例} % 图片标题
        \label{fig:example} % 图片标签
    \end{figure}

    \subsubsection{计算辐射量}
    
    \subsubsection{结果的检验}
    \subsubsection{问题的结论}
    \subsection{问题二模型的建立与求解}
    
    \subsubsection{某某模型的建立与求解}
    \subsection{问题三模型的建立与求解}
    \subsubsection{模型的求解}
    
    \subsection{问题四模型的建立与求解}
    \subsubsection{问题四模型的建立}
    \subsubsection{问题四模型的求解}
            
    \section{模型评价与改进}
    \subsection{模型优点}
    \begin{enumerate}[(1)]
        \item 对
        \item 采用算法
    \end{enumerate}
    \subsection{模型缺点}
    \begin{enumerate}[(1)]
        \item 问题一中
        \item 结果
    \end{enumerate}

    \begin{thebibliography}{9} % 参考文献
		\bibitem{bib:8}何晓群.多元统计分析.北京:中国人民大学出版社,2012.
		\bibitem{bib:9}徐维超. 相关系数研究综述[J]. 广东工业大学学报,2012,29(3):12-17.
    \end{thebibliography}
    
    \newpage
    \section{附录}
    %插入代码内容
    \begin{lstlisting}
    \end{lstlisting}
\end{document}       